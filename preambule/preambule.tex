\usepackage{amsmath,amsfonts,amssymb}%extensions de l'ams pour les math�matiques
\usepackage{amsthm}
\usepackage{shorttoc}%pour la r�alisation d'un sommaire
\usepackage{tikz}
\usepackage{graphicx}%pour ins�rer images et pdf entre autres
\usepackage{subfig}
\usepackage{epsfig}
\usepackage{stackrel}
\usepackage{lscape}
\usepackage{longtable}
\usepackage{color, colortbl}
\definecolor{LightCyan}{rgb}{0.88,1,1}
\definecolor{LavenderBlush}{rgb}{0.93,0.88,0.9}

\usepackage[ruled, vlined, linesnumbered]{algorithm2e}
        \graphicspath{{images/}}%pour sp�cifier le chemin d'acc�s aux images
\usepackage[left=3.5cm,right=2.5cm,top=4cm,bottom=4cm]{geometry}%r�glages des marges du document selon vos pr�f�rences ou celles de votre �tablissement
\usepackage[Lenny]{fncychap}%pour de jolis titres de chapitres voir la doc pour d'autres styles.
\usepackage{fancyhdr}%pour les en-t�tes et pieds de pages
       \setlength{\headheight}{14.2pt}% hauteur de l'en-t�te
%%%%%%%%%%%%%%%%%%%style front%%%%%%%%%%%%%%%%%%%%%%%%%%%%%%%%%%%%%%%%%
       \fancypagestyle{front}{%
               \fancyhf{}%on vide les en-t�tes
               \fancyfoot[C]{ \thepage}%
               \renewcommand{\headrulewidth}{0pt}%trait horizontal pour l'en-t�te
               \renewcommand{\footrulewidth}{0.4pt}%trait horizontal pour les pieds de pages
               }
%%%%%%%%%%%%%%%%%%%%style main%%%%%%%%%%%%%%%%%%%%%%%%%%%%%%%%%%%%
       \fancypagestyle{main}{%
               \fancyhf{}
               \renewcommand{\chaptermark}[1]{\markboth{\chaptername\ \thechapter.\ ##1}{}}% red�fintion pour avoir ici les titres des chapitres des sections en minuscules
               \renewcommand{\sectionmark}[1]{\markright{\thesection\ ##1}}
               \fancyhead[c]{}
               \fancyhead[RO,LE]{\rightmark}%
               \fancyhead[LO,RE]{\leftmark}
               \fancyfoot[C]{}
               \fancyfoot[RO,LE]{\thepage}%
               \fancyfoot[LO,RE]{Ph.D. Thesis}
               }
%%%%%%%%%%%%%%%%%%%%style back%%%%%%%%%%%%%%%%%%%%%%%%%%%%%%%%%%%%%%%%%  
       \fancypagestyle{back}{%
               \fancyhf{}%on vide les en-t�tes
               \fancyfoot[C]{ \thepage}%
               \renewcommand{\headrulewidth}{0pt}%trait horizontal pour l'en-t�te
               \renewcommand{\footrulewidth}{0.4pt}%trait horizontal pour les pieds de pages
               }
%%%%%%%%%%%%%%%%%%%%%%%%%%%%%index%%%%%%%%%%%%%%%%%%%%%%%%%%%%%%%%%%%%%%%
\usepackage{makeidx}
\makeindex
\usepackage[utf8]{inputenc}
\usepackage[T1]{fontenc}
\usepackage[english]{babel}%pour un document en fran�ais
\usepackage{listings}%pour ins�rer du code source
\usepackage{hyperref}%rend actif les liens, r�f�rences crois�es, toc�
               \hypersetup{colorlinks,%
               citecolor=black,%
               filecolor=black,%
               linkcolor=black,%
               urlcolor=black}
%%%%%%%%%%%%%%%%%%%%%%%%%%%%%biblio%%%%%%%%%%%%%%%%%%%%%%%%%%%%%%%%%%%%%%
%\usepackage[stype=named]{biblatex}
%\addbibresource{bibliographie/biblio.bib}% pour indiquer o� se trouve notre .bib
%\usepackage{csquotes}% pour la gestion des guillemets fran�ais.
% %%%%%%%%%%%%%%%%%%%%%%%%%%%%%%glossaire%%%%%%%%%%%%%%%%%%%%%%%%%%%%%%%%%%%
 \usepackage{glossaries}
 \usepackage{frcursive}
 \makeglossaries         
% %%%%%%%%%%%%%%%%%%%%%%%%%%%%%liste des abr�viations%%%%%%%%%%%%%%               
\usepackage[english]{nomencl}
\makenomenclature
\renewcommand{\nomname}{Liste des abreviations, des sigles et des symboles}
\makeatletter

%%%%%%%%%%%%%%%%%%%%%%%%%%%%%%D�finition%%%%%%%%%%%%%%%%%%%%%%%%%%%%%%%%%%%%

% \newenvironment{definition}{%
%   \itshape
% }{%
%   % Rien
% }
\newtheorem{definition}{Definition}
\newtheorem{theorem}{Theorem}
%\onehalfspacing

\newenvironment{abstract}
{
\em
\paragraph{Abstract: }
}
{
\\
}