\chapter*{Introduction}

\section*{Context}

In almost every scientific field, measurements are performed over time. These observations lead to a collection
of organized data called time series. Today time series data are being generated at an unprecedented
speed from almost every application domain, e.g.: 

\begin{itemize}
\item In astronomy, telescopes scan the sky and capture light rays that are used in the study of the universe. In Large Synoptic Survey Telescope (LSST) project \cite{lsst}, telescopes will capture the electromagnetic radiation of the sky during ten years to calculate the acceleration of the expansion of the universe. This will result in an astronomical catalogs of time series.
\item In paleoecology, ...
\item In medicine, the analysis is electrocardiogram is used to prevent heart attacks[]. Those electrocardiograms are long time series obtained by recording the electrical activity of the heart over a period.  
\item In biomechanics, the study of human locomotion is perform using sensors that record the efforts performed and the movements of the body during the locomotion.
\end{itemize}

As a consequence, in the last decade there has been a dramatically increasing amount of interest in querying and mining such data.

\section*{Issues}



Time-series data mining unveils numerous facets of complexity. The most prominent
problems arise from the uncertainty contained in time series data,  the difficulty of
defining a form of similarity measure based on human perception, and the high dimensionality of time-series data. These constraints show us that three major issues are involved.


\begin{itemize}
\item Uncertainty. How to compare the shape of time series without knowing their exact value? How to measure the impact of uncertainty contained in time series or to reduce the adverse effects of uncertainty? 
\item Similarity measurement. How can any pair of time-series be distinguished or
matched? How can an intuitive distance between two series be formalized? This
measure should establish a notion of similarity based on perceptual criteria, thus
allowing the recognition of perceptually similar objects even though they are not
mathematically identical.  
\item Data representation. How can the fundamental shape characteristics of a time-series
be represented? What invariance properties should the representation satisfy? A
representation technique should derive the notion of shape by reducing the dimensionality
of data while retaining its essential characteristics.
\end{itemize}

The aim of our work is to propose algorithm to deal with thoses caracteristics of time series.

\section*{Context of the thesis}
This thesis apprehends these scientific questions from a data mining point of view, within the framework of the analysis of time series coming from Manual Wheelchair locomotion. Also, even if the issues addressed are not limited to the field of biomechanics time series and concern other areas of applications, this thesis will deal with the analysis of time series coming from Ergometer Wheelchair FRET-2.


For improving the mobility of persons confined to manual wheelchairs, it is necessary to be able to "assess" people in their daily environment, and a field ergometer wheelchair (FRET-1) has been designed and manufactured for this purpose []. , ]. This ergometer is equipped with a moment sensor that measures the forces applied to the handrails as well as the acceleration and movement of the FRET-1 [, ]. It, therefore, makes it possible to measure and calculate a large number of the mechanical parameters of manual wheelchair locomotion.


However, the time series produced by this moment sensor have specific characteristics: 
\begin{itemize}
\item they are long because of the acquisition frequency of the sensor (between 80 and 100 Hz),
\item they are cyclic; these cycles come from the cyclical character of the locomotion in Manual Wheelchair which consists of a succession of period of pushing and freewheeling,
\item they are uncertain, this uncertainty is observed during the calibration of the sensor.
\end{itemize}
  

Our work consists of proposing algorithms to extract relevant information from these time series while taking into account their characteristics. The methods developed in this work have the aim to assist practitioners for the analysis of Manual Wheelchair locomotion; then, special attention will be given to the readability and ease of interpretation of the results provided by them.


\section*{Plan}

The thesis is organised as follow
\begin{itemize}
\item \textbf{Chapter\,1} present the state of art
\item \textbf{Chapter\,2}  Dynamic Time Warping (DTW) is a time series alignment algorithm that is often used because it
 considers that it exits small distortions between time series during their alignment.  However, DTW
 sometimes produces pathological alignments that occur when, during the comparison of two time series
 X and Y, one data point of the time series X is compared to a large subsequence of data points of Y.
 In this paper, we demonstrate that to compress time series using Piecewise Aggregate Approximation
 (PAA) is a simple strategy that greatly increases the quality of the alignment with DTW this is
 particularly true for synthetic data sets.
\item \textbf{Chapter\,3} The abstract.
\item \textbf{Chapter\,4} The analysis of cyclic time series from biomechanics is based on the
comparison of the properties of their cycles. As usual algorithms of time 
series classification ignore this particularity, we propose
a symbolic representation of cyclic time series based on the properties
of cycles, named SAX-P. The resulting character strings can be compared
using the Dynamic Time Warping distance. The application of SAX-P
to propulsive moments of three subjects (S1, S2, S3) moving in Manual
Wheelchair highlight the asymmetry of their propulsion. The symbolic representation 
SAX-P facilitates the reading of
the cyclic time series and the clinical interpretation of the classification results. 
\end{itemize}


