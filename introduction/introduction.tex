%\chapter*{\textbf{General introduction} }

\section*{Context}

In almost every scientific field, measurements are performed over time. These observations lead to a collection
of organized data called time series. Today time series data are being generated at an unprecedented
speed from almost every application domain, e.g.: 

\begin{itemize}
\item In astronomy, telescopes scan the sky and capture light rays that are used in the study of the universe. In Large Synoptic Survey Telescope (LSST) project \cite{lsst}, telescopes will capture the electromagnetic radiation of the sky during ten years to calculate the acceleration of the expansion of the universe. This will result in an astronomical catalog of time series.
\item In paleoecology, scientists study the evolution of living animal and plant species in the past. To do this, they extract cores from the soil and look for the presence of fossils at each depth, thus creating time series representing the growth or decline in the size of fauna and flora populations over time \cite{lonlacfouille}. 
\item In medicine, the analysis is electrocardiogram is used to prevent heart attacks \cite{ding2011key}. Those electrocardiograms are long time series obtained by recording the electrical activity of the heart over a period.  
\item In biomechanics, the study of human locomotion is perform using sensors that record the efforts performed and the movements of the body during the locomotion.
\end{itemize}

As a consequence, in the last decade there has been a dramatically increasing amount of interest in querying and mining such data.

\section*{Issues}



Time series data mining unveils numerous facets of complexity. The most prominent
problem arise from the uncertainty contained in time series data,  the difficulty of
defining a form of similarity measure based on human perception, and the high dimensionality of time series data. These constraints show us that three major issues are involved :


\begin{itemize}
\item Data representation. How can the fundamental shape characteristics of a time series
be represented? What invariance properties should the representation satisfy? A
representation technique should derive the notion of shape by reducing the dimensionality
of data while retaining its essential characteristics.

\item Similarity measurement. How can any pair of time series be distinguished or
matched? How can an intuitive distance between two series be formalized? This
measure should establish a notion of similarity based on perceptual criteria, thus
allowing the recognition of perceptually similar objects even though they are not
mathematically identical.  

\item Uncertainty. How to compare the shape of time series without knowing their exact value? How to measure the impact of uncertainty contained in time series or to reduce the adverse effects of uncertainty? 
\end{itemize}

The aim of our work is to propose algorithms to deal with thoses caracteristics of time series.

\section*{Context of the thesis}
This thesis apprehends these scientific questions from a data mining point of view, within the framework of the analysis of time series coming from Manual Wheelchair (MWC) locomotion. Also, even if the issues addressed are not limited to the field of biomechanics time series and concern other areas of applications, this thesis will deal with the analysis of time series coming from the wheelchair ergometer FRET-2.


For improving the mobility of persons confined to manual wheelchairs, it is necessary to be able to assess people in their daily environment. For this purpose, a field ergometer wheelchair (FRET-1) has been designed and manufactured  \cite{dabonneville2005self}. This ergometer is equipped with  moment sensors that measures the forces applied to the handrims as well as the movement of the FRET-1 \cite{couetard2000}. It, therefore, makes it possible to measure and calculate a large number of the mechanical parameters of manual wheelchair locomotion.


However, the time series produced by this moment sensor have specific characteristics: 
\begin{itemize}
\item they are long because of the acquisition frequency of the sensor (between 80 and 100 Hz),
\item they are cyclic; these cycles come from the cyclical character of the locomotion in Manual Wheelchair, which consists of a succession of push phases and ecovery (or freewheeling) phases,
\item they are uncertain, this uncertainty is observed during the calibration of the sensor.
\end{itemize}
  

Our work consists of proposing algorithms to extract relevant information from these time series while taking into account their characteristics. The methods developed in this work have the aim to assist practitioners for the analysis of Manual Wheelchair locomotion; then, special attention will be given to the readability and ease of interpretation of the results provided by the  algorithms.


\section*{Plan}

The thesis is organised as follows:
\begin{itemize}
\item \textbf{Chapter\,\ref{locomotion_analysis}} presents the main issues related to manual wheelchair locomotion analysis, a literature review of physical and computer models and tools for assessing wheelchair locomotion. The objective of this chapter is to show that over the years manual wheelchair locomotion analysis tools have been upgraded and have allowed the construction of a manual field ergometer wheelchair called FRET of which data are the subject of our analysis. It also presents works from the literature conducted in the fields of biomechanics and computer science with the aim of analyzing manual wheelchair locomotion, which will enable the reader to position our work with what already exists.


\item \textbf{Chapter\,\ref{kdd}} explains existing models in the field of time series processing that could be useful for the analysis of  Wheelchair locomotion. Thus, chapter \ref{kdd} presents strategies for the preprocessing of time series (e.g. noise reduction, length reduction), their comparisons, their exploitation through visualization, classification, clustering or prediction.
 
 
\item \textbf{Chapter\,\ref{fdtw}} introduces an algorithm called FDTW, which aims to reduce the length of time series while preserving the information it contains. Its operating principle is based on that of GRASP, but it is original in that it defines its global search strategy. Experiments conducted on a classification task have shown that compression does not alter classification performance. 

\item \textbf{Chapter\, \ref{fots}} proposes a novel framework for uncertain time series clustering, which is based on the use of a clustering algorithm (UShapelet), and on the use of a dissimilarity function (FOTS)  the both are robust to the presence of uncertainty in time series. We tested this clustering strategy on 17 data sets from the literature, which allowed us to observe an improvement in the quality of the obtained results.


\item \textbf{Chapter\,\ref{chapter_saxp}} presents a novel symbolic representation of cyclic time series based on cycle properties, which we use for the analysis of cyclic time series issued by human locomotion. This symbolic representation  facilitates the visualization and evaluation of cyclic time series.

\item \textbf{Chapter\,\ref{application}} gives some applications of the proposed algorithms to data from manual wheelchair locomotion. The results allowed us to measure the asymmetry of wheelchair locomotion and to establish that this asymmetry decreases with years of practice. We have also observed that the propulsion capabilities of wheelchair users with similar levels of spinal cord injury may differ. Also, wheelchair users' propulsion technique evolves, but this evolution varies according to the subject. These last two results highlight the importance of monitoring manual wheelchair locomotion using measuring instruments.  
\end{itemize}


