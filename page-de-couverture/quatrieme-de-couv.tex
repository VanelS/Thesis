\clearpage
\ifodd\thepage\hbox{}\newpage\else\fi%si page paire ou impaire
\thispagestyle{empty}\parindent=0pt
\addcontentsline{toc}{chapter}{Résumé}
{\Large \textbf{Extraction de connaissances de séries temporelles cycliques et incertaines : application à l'analyse de la locomotion en fauteuil roulant manuel}}
{\large \textbf{}}
\hrulefill%trace un trait horizontal
\begin{center}
{\Large \textbf{Résumé}}
\end{center}
L'évaluation des capacités motrices des utilisateurs de Fauteuil roulant manuel est souvent subjective, car elle se base sur l'avis d'un expert. C'est pourquoi, un fauteuil roulant manuel ergomètre de terrain a été fabriqué. Il permet d'enregistrer les efforts effectués par les utilisateurs de fauteuil roulant manuel pendant leur déplacement. Les mesures ainsi effectuées sont des séries temporelles ayant les caractéristiques spécifiques suivantes : elles sont longues, incertaines et cycliques. En nous appuyant sur ces mesures ainsi que sur leurs propriétés, l'objectif de cette thèse est d'effectuer une analyse objective de la locomotion en fauteuil roulant manuel. À cet effet, nous proposons trois modèles. Le premier modèle est une heuristique permettant de trouver le nombre judicieux de segment à considérer pour la compression des séries temporelles avec l'algorithme d'approximation par morceau, tout en concevant l'information contenue dans les séries temporelles. Si le principe de fonction de cette heuristique est similaire à celui de la recherche gloutonne randomisée, elle a la particularité de proposer une stratégie spécifique de recherche globale. Le deuxième modèle est une mesure de similarité qui permet de capturer la structure fondamentale des séries temporelles et qui est robuste à la présence d'incertitude. Cette mesure de similarité est basée sur la comparaison à l'aide de la norme de Frobenius des vecteurs propres des matrices d'autocorrélation des séries temporelles. Le troisième modèle est une représentation symbolique de séries temporelle cycliques basée sur les propriétés de cycles qui utilise un algorithme de segmentation des séries temporelles cycliques en cycle et un algorithme de classification non supervisée pour comparer les cycles en fonction de leurs propriétés. Cette représentation symbolique permet une meilleure visualisation et une meilleure analyse basée sur les propriétés des cycles des séries temporelles cycliques. Nos modèles permettent de mettre en évidence le caractère asymétrique de la locomotion en fauteuil roulant manuel et d'établir que l'asymétrie de la locomotion diminue avec les années de pratique. Ils permettent également d'évaluer de manière objective et intelligible les capacités motrices des utilisateurs de fauteuil roulant manuel.

\vspace*{\stretch{1}}
\hrulefill%trace un trait horizontal
{\Large \textbf{Séries temporelles, compression, comparaison, représentation }}
\hrulefill%trace un trait horizontal
\vspace*{\stretch{1}}


\newpage

\selectlanguage{english}
\addcontentsline{toc}{chapter}{Abstract}
{\Large \textbf{    Extraction of knowledge from cyclical and uncertain time series: application to Manual Wheelchair locomotion analysis}}
{\large \textbf{}}
\hrulefill%trace un trait horizontal
\begin{center}
{\Large \textbf{Abstract}}
\end{center}

The assessment of the motor skills of manual wheelchair users is often subjective because it is based on expert opinion. Therefore, a manual field ergometer wheelchair was conceived and constructed. It records the efforts made by manual wheelchair users during their locomotion. The measurements made are time series with the following specific characteristics: they are long, uncertain and cyclical. Based on these measurements and their properties, the objective of this thesis is to perform an objective analysis of manual wheelchair locomotion. To this, we proposed three models. The first model is a heuristic to find the appropriate number of segments to consider for time series compression with the piece aggregate approximation algorithm while keeping the information they contained. If the principle of this heuristic is similar to that of greedy randomized adaptive search, it has the particularity of proposing a specific global search strategy. The second model is a similarity measure that captures the fundamental structure of time series and is robust to the presence of uncertainty. This similarity measure is based on the comparison using the Frobenius norm of eigenvectors of time series autocorrelation matrices. The third model is a symbolic representation of cyclic time series based on cycle properties that utilises a segmentation algorithm of cyclic time series  in cycles and an unsupervised classification algorithm to compare cycles according to their properties. This symbolic representation allows better visualization and analysis based on the cycle properties of cyclic time series. Our models highlight the asymmetrical nature of manual wheelchair locomotion and establish that the asymmetry of locomotion decreases with years of practice. They also provide an objective and intelligible assessment of the motor abilities of manual wheelchair users.

\vspace*{\stretch{1}}
\hrulefill%trace un trait horizontal
{\Large \textbf{Time series, compression, comparison, representation}}
\hrulefill%trace un trait horizontal
%\selectlanguage{frensh}
\vspace*{\stretch{1}}
\hrulefill
{\Large \textbf{University Clermont Auvergne} }
\hfill\includegraphics[scale=0.5]{./images/logoLimos.png} 
\hrulefill