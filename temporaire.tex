
%\subsection*{Abstract}
%{\em
%To reduce the length of time series, the piece wise aggregate approximation method  divides
%a time serie into fixed-length segments, each segment is then replaced by the average of its
%points. The users of this method want to have the most compact representation, but if the 
%considered number of segments is too small, this causes a loss of information which can lead to poor
%results in a classification task. How then to choose the number of segments to be considered for a
%compact representation of a time series? In this article, we define a lowerbound for the number
%of segments to be considered and propose an algorithm that chooses the number of segments that
%minimizes the mean squared error. The results of experiments conducted on 85 datasets show that this compact representation significantly reduces the length
%of time series and reduces the classification error in many time series datasets. This process
%reduces the storage space and runtime of time series. It also allows to choose
%the appropriate number of segments to be used for symbolic representations of time series such as
%SAX, ESAX, 1d-SAX, SAX-TD.} 
%\subsection*{Keywords} Time series, dimensionality reduction, PAA, mean
%squared error.

