\chapter*{\textbf{General conclusion and Future works}}
\section*{Aims}
Our primary objective throughout this work was to analyze manual wheelchair locomotion using measurements made by the sensors when using the wheelchair.  This primary objective is divided into three specific goals: how to pre-treat time series to reduce their length and therefore their processing time, how to take into account the existence of uncertainty in the time series during their analysis and finally how to base the exploitation of time series on the cycles that constitute them. Each of these specific objectives has resulted in proposed models.
\section*{Summary of contributions}
\subsection*{Reduce the length of time series with FDTW}
We proposed a heuristic named FDTW that find a suitable parameter to use with the piecewise aggregate approximation algorithm with the aim to reduce the length of time series for classification purpose. This heuristic is based on Greedy Randomized  Adaptative Search Procedure but defines a specific global search strategy. Extensive experimentation has been run out and shows that the compression with FDTW allows reducing the length of time series while keeping their main shape. Moreover, the compression with FDTW can enhance the accuracy of classification because it will enable avoiding pathological alignment with Dynamic Time Warping algorithm this amelioration is particularly perceptible with synthetic datasets.
\subsection*{Dealing with uncertainty using FOTS score}
We introduce a novel dissimilarity score for the comparison of time series named FOTS for Frobenius Correlation for Time series uShapelet discovery. This dissimilarity score is based on local correlation and allows to capture internal properties of the time series while being robust to uncertainty because its computation is based on the comparison of eigenvectors of the autocorrelation matrices of time series. This score has been used for clustering purpose with UShapelet clustering algorithm an shows a significant improvement of the quality of clustering according to the rand Index.
\subsection*{Taking into accounts cycles with a symbolic representation}
Time series coming from wheelchair locomotion are cyclic due to the cyclic aspect of the wheelchair locomotion. The analysis of the wheelchair locomotion is based on those cycles, but none of the data mining models of the literature consider this aspect. We then proposed a symbolic representation of cyclic time series based on the properties of cycles that allow better visualization of the data and a better comprehension of the results obtained after the data mining process. We use this symbolic representation for the analysis of the wheelchair locomotion of eleven users, and this symbolic representation allows establishing that the wheelchair locomotion is asymmetric but this asymmetry get lower and lower with the years' of practice. This symbolic representation also allows a more precise evaluation of motor capabilities of manual wheelchair mainly based on the effort measure during their use of the manual wheelchair.
\section*{Future works and prospects}
\subsection*{Multidimensionality for a better characterization of the propulsion technic}
The analyses in this thesis are based on the Z moment of the wheels of the manual wheelchair. However, several measurements, including seat, back and footrest forces, were taken during the locomotion of the manual wheelchair users. Considering these signals could allow a better analysis of the subjects' movement and therefore suggests that we propose multidimensional data mining models that would simultaneously take into account all the measurements and their characteristics, namely their length, the presence of uncertainty, and the cyclical nature of specific measures.
\subsection*{Fuzziness for a more realistic categorization}
As we saw in Chapter 6 with subject S05, a subject can be very likely to belong to two or more groups. It would, therefore, be wise to associate each assignment with a degree of trust about the subject's group member. This data mining amounts to considering fuzzy approaches in the analysis of time series from manual wheelchair locomotion.